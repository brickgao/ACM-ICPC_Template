\subsection{最大流 Dinic}
用邻间表实现。
    \begin{lstlisting}[language=c++]
struct Edge {
    int u, v, cap;
};
int dep[1100], ptr[1100], n, s, t;
vector<Edge> e;
vector<int> g[1100];

inline void init() {
    for(int i = 0; i < n; i ++)
        g[i].clear();
    e.clear();
}

inline void add_edge(int u, int v, int cap) {
    Edge e1 = {u, v, cap};
    Edge e2 = {v, u, 0};
    g[u].push_back(e.size());
    e.push_back(e1);
    g[v].push_back(e.size());
    e.push_back(e2);
}

bool bfs() {
    queue<int> q;
    memset(dep, -1, sizeof(dep));
    dep[s] = 0;
    q.push(s);
    while(!q.empty()) {
        int u = q.front();
        q.pop();
        for(int i = 0; i < (int)g[u].size(); i ++) {
            int id = g[u][i];
            int v = e[id].v;
            if(dep[v] == -1 && e[id].cap) {
                dep[v] = dep[u] + 1;
                q.push(v);
            }
        }
    }
    return dep[t] != -1;
}

int dfs(int u, int flow) {
    if(!flow) return 0;
    if(u == t) return flow;
    for(; ptr[u] < (int)g[u].size(); ptr[u] ++) {
        int id = g[u][ptr[u]];
        int v = e[id].v;
        if(dep[u] + 1 != dep[v])
            continue;
        int add = dfs(v, min(flow, e[id].cap));
        if(add) {
            e[id].cap -= add;
            e[id^1].cap += add;
            return add;
        }
    }
    return 0;
}

int dinic() {
    int maxflow = 0;
    while(bfs()) {
        memset(ptr, 0, sizeof(ptr));
        while(true) {
            int minflow = dfs(s, inf);
            if(minflow)
                maxflow += minflow;
            else break;
        }
    }
    return maxflow;
}
    \end{lstlisting}
用链表实现。
    \begin{lstlisting}[language=c++]
bool bfs()
{
	while(!que.empty()) que.pop();
	memset(dist,-1,sizeof(dist));
	dist[s]=0;
	que.push(s);
	while(!que.empty())	
	{
		int p=que.front();
		que.pop();
		for(int x=last[p];x!=0;x=pre[x])
		{
			if(len[x]>0 && dist[son[x]]<0)
			{
				dist[son[x]]=dist[p]+1;
				que.push(son[x]);	
			}	
		}
	}
	return dist[t]>0;
}
int find(int k,int flow)
{
	if(k==t) return flow;
	int f=0,tmp;
	for(int x=last[k];x!=0 && f<flow;x=pre[x])
	{
		if(len[x]>0 && dist[son[x]]==dist[k]+1)
		{
			tmp=find(son[x],min(flow-f,len[x]));
			if(tmp==0) dist[son[x]]=-1;
			else 
			{
				f+=tmp;
				len[x]-=tmp;
				if(x&1) len[x+1]+=tmp;
				else len[x-1]+=tmp;	
			}	
		}	
	}	
	return f;
}
int main()
{
	int tmp,ans=0;
	while(bfs())
		while(tmp=find(s,914990825))
			ans+=tmp;
}
    \end{lstlisting}
