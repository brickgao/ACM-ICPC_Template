\subsection{最大流Dinic}
    \begin{lstlisting}[language=c++]
int n, s, t, dis[200], pre[200];
struct Edge {
    int u, v, cap, cost;
};
vector<int> g[200];
vector<Edge> e;

inline void init() {
    for(int i = 0; i < n; i ++)
        g[i].clear();
    e.clear();
}

inline void add_edge(int u, int v, int cap, int cost) {
    Edge e1 = {u, v, cap, cost};
    Edge e2 = {v, u, 0, -cost};
    g[u].pb(e.size());
    e.pb(e1);
    g[v].pb(e.size());
    e.pb(e2);
}

bool spfa() {
    bool inq[200];
    for(int i = 0; i < n; i ++) {
        pre[i] = -1;
        dis[i] = inf;
        inq[i] = false;
    }
    queue<int> q;
    dis[s] = 0;
    inq[s] = true;
    q.push(s);
    while(!q.empty()) {
        int u = q.front();
        q.pop();
        inq[u] = false;
        for(int i = 0; i < (int)g[u].size(); i ++) {
            int id = g[u][i];
            int v = e[id].v;
            if(e[id].cap && get_min(dis[v], dis[u] + e[id].cost)) {
                pre[v] = id;
                if(!inq[v]) {
                    inq[v] = true;
                    q.push(v);
                }
            }
        }
    }
    return dis[t] < inf;
}

pair<int, int> mcmf() {
    int maxflow = 0;
    int mincost = 0;
    while(spfa()) {
        int minflow = inf;
        for(int i = pre[t]; i != -1; i = pre[e[i].u]) 
            get_min(minflow, e[i].cap);
        for(int i = pre[t]; i != -1; i = pre[e[i].u]) {
            e[i].cap -= minflow;
            e[i^1].cap += minflow;
        }
        maxflow += minflow;
        mincost += minflow * dis[t];
    }
    return make_pair(mincost, maxflow);
}

int main() {
    return 0;
}
    \end{lstlisting}
