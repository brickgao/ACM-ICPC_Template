\subsection{上下界流}
建图
    \begin{lstlisting}[language=c++]
for(int i = 1; i <= m;i++)
{
	scanf("%d %d %d %d",&a,&b,&c,&d);   
    add(a,b,d-c);
    add(b,a,0);
    du[a]-=c;
    du[b]+=c;
}
    \end{lstlisting}
处理有源汇有上下界最大流问题是:
1.构造附加网络   (添加 t->s 容量inf 添加ss和tt  ss->v(du[v]>0) v->tt(du[v]<0)  c(u,v) = 上界-下界  du[i]为所有流入i的下界减去所有流出i的下界
2.对ss、tt求最大流(ss、tt满流则有解)
3.若有解,对s、t求最大流
 
而有源汇有上下界最小流问题则是:
1.构造附加网络(不添加[t,s]边)
2.对ss、tt求最大流
3.添加[t,s]边
4.对ss、tt求最大流
5.若ss、tt满流,则[t,s]的流量就是最小流
