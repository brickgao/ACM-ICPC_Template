\subsection{Polya定理}
题意为:给你c种颜色的珠子,和一个长度为s(c,s<32)的项链,用这c种珠子串成这个项链,项链可以旋转和翻转,经过旋转和翻转所得的项链视为同一种项链,现在告诉你颜色总数c和项链的长度s,求共能组成几条不同的项链。
1.旋转置换
    依次顺时针旋转1 ~ n个,循环个数为gcd(i, n)
2.翻转置换
    当n为偶数时,分两种情况,一种是中心轴在两个对称对象上,则循环个数为n/2+1,另一种是对称轴两边分别有n/2个对象,则循环个数为n/2;
    当n为奇数时,对称轴就只能在一个对象上,则循环个数为n/2+1;
    \begin{lstlisting}[language=c++]
int c, s, ans;

int gcd(int a, int b) {
    if(b == 0)  return a;
    return gcd(b, a % b);
}

void solve() {
    int t = 0;
    ans = 0;
    for(int i = 1; i <= s; i ++) {
        ans += pow(c, gcd(i, s));
        t ++;
    }
    if(s % 2 == 0) {
        for(int i = 0; i < s / 2; i ++) {
            ans += pow(c, s / 2);
            t ++;
        }
        for(int i = 0; i < s / 2; i ++) {
            ans += pow(c, s / 2 + 1);
            t ++;
        }
    }
    else {
        for(int i = 0; i < s; i ++) {
            ans += pow(c, s / 2 + 1);
            t ++;
        }
    }
    ans /= t;
}

int main() {
    while(scanf("%d%d", &c, &s) != EOF) {
        if(c == 0 && s == 0) {
            return 0;
        }
        solve();
        printf("%d\n", ans);
    }
    return 0;
}
    \end{lstlisting}
给出两个整数n和p,代表n个珠子,n种颜色,要求不同的项链数,并对结果mod(p)处理,置换只有旋转一种形式。
这道题代表了一类题目的优化
裸的算法是 ∑n(gcd(n,i)) 1<i<=n
复杂度过高,进行优化。
置换群种循环的个数L = n / gcd(n, i)
因为如果L | n, 则有n / L | n
则环的长度L的范围是1 ~ sqrt(L)
令a = gcd(n, i), 设i = at
则只有i与t互质的时候,gcd(i, t) = a
则最后可以优化为 ∑(Φ(i) * n \^ (i)) \% p

    \begin{lstlisting}[language=c++]
long long ans;
int t, num, n, p;
int isprime[50001];
int prime[8001];

void getprime() {
    num = 0;
    for(int i = 2; i <= 50000; i ++) {
        if(!isprime[i]) {
            prime[num ++] = i;
            for(int j = 1; j * i <= 50000; j ++) {
                isprime[i * j] = 1;
            }
        }
    }
}

int euler(int x) {
    int res = x;
    for(int i = 0; i < num && prime[i] * prime[i] <= x; i++) {
        if(x % prime[i] == 0) {
            res = res / prime[i] * (prime[i] - 1);
            while(x % prime[i] == 0) {
                x /= prime[i];
            }
        }
    }
    if(x > 1) res = res / x * (x - 1);
    return res;
}

LL quickpow(LL m , LL n , LL k) {
    LL tmp = 1; 
    m %= k;
    while(n) { 
        if(n & 1)
            tmp = (tmp * m) % k; 
        m = (m * m) % k;
        n >>= 1;
    } 
    return tmp;
} 

int main() {
    getprime();
    scanf("%d", &t);
    while(t --) {
        ans = 0;
        scanf("%d%d", &n, &p);
        for(int i = 1; i < sqrt(n); i ++) {
            if(n % i == 0) {
                ans = (ans + euler(i) % p * quickpow(n, n / i - 1, p) + euler(n / i) % p * quickpow(n, i - 1, p)) % p;
                //cout << quickpow(n,n-1,p) << " " << quickpow(n, n / i - 1, p) << " " << euler(i) << " " << euler(n / i) << endl;
            }
        }
        if((int)sqrt(n) * (int)sqrt(n) == n) {
            ans = (ans + quickpow(n, sqrt(n) - 1, p) * (euler(sqrt(n)) % p)) % p;
        }
        ans %= p;
        cout << ans << endl;
    }
    return 0;
}
    \end{lstlisting}
