\subsection{凸多边形面积并}
    \begin{lstlisting}[language=c++]
const double eps=1e-9;
const int N=150;
int n,m=3;
inline int sgn(double x)
{
    if(fabs(x)<eps)return 0;
    return x>0?1:-1;
}
struct L
{
    double k,b;
    L(){}
    L(double _k,double _b):k(_k),b(_b){}
    bool operator<(const L &a)const
    {
        if(!sgn(k-a.k))return b<a.b;
        return sgn(k-a.k)<0;
    }
    bool operator==(const L &a)const 
    {
        return !sgn(k-a.k)&&!sgn(b-a.b);
    }
};
struct P
{
    double x,y;
    inline void input()
    {
        scanf("%lf%lf",&x,&y);
    }
    P(){}
    P(double _x,double _y):x(_x),y(_y){}
    P operator-(const P b)const
    {
        return P(x-b.x,y-b.y);
    }
    double operator*(const P b)const
    {
        return x*b.y-b.x*y;
    }
    bool operator<(const P &a)const
    {
        if(!sgn(x-a.x))return sgn(y-a.y)<0;
        return sgn(x-a.x)<0;
    }
    bool operator==(const P &a)const 
    {
        return !sgn(x-a.x)&&!sgn(y-a.y);
    }
};
P cp;
struct SEG
{
    P a,b;
    SEG(){}
    SEG(P _a,P _b):a(_a),b(_b){}
};
vector<SEG>seg1,seg2;
L line[10*N];
struct POLY
{
    P p[4];
    P&operator[](int x){return p[x];}
    bool check()
    {
        p[m]=p[0];
        double res=0;
        int i;
        for(i=0;i<m;i++)res+=p[i]*p[i+1];
        if(!sgn(res))return false;
        if(sgn(res)<0)reverse(p,p+m+1);
        return true;
    }
}tr[N];
void init()
{
    int i,j,k=0;
    scanf("%d",&n);
    for(i=1;i<=n;i++)
    {
        for(j=0;j<m;j++)
        tr[i][j].input();
        if(tr[i].check())tr[++k]=tr[i];
    }
    n=k;
}
L get_L(const P a,const P b)
{
    P c=b-a;
    double k=c.y/c.x;
    double bb=b.y-b.x*k;
    return L(k,bb);
}
bool get_cut(const L f,const P a,const P b)
{
    double tmp=f.k*(a.x-b.x)-(a.y-b.y);
    if(!sgn(tmp))return false;
    cp.x=(a*b-(a.x-b.x)*f.b)/tmp;
    double sx=min(a.x,b.x),ex=max(a.x,b.x);
    cp.y=cp.x*f.k+f.b;
    double sy=min(a.y,b.y),ey=max(a.y,b.y);
    return (sgn(cp.x-sx)>=0)&&(sgn(cp.x-ex)<=0)&&(sgn(cp.y-sy)>=0)&&(sgn(cp.y-ey)<=0);
}
double cal(const vector<SEG>seg)
{
    int i,k=0,len=seg.size(),j;
    bool touch;
    double res=0;
    for(i=0;i<len;i++)line[++k]=get_L(seg[i].a,seg[i].b);
    sort(line+1,line+1+k);
    len=unique(line+1,line+1+k)-line-1;
    for(i=1;i<=len;i++)
    {
        vector<pair<double,int>>mark;
        for(j=1;j<=n;j++)
        {
            vector<P>cut;
            touch=false;
            for(k=0;k<m;k++)
            if(line[i]==get_L(tr[j][k],tr[j][k+1]))
            {
                touch=true;
                break;
            }
            if(touch)continue;
            for(k=0;k<m;k++)
            if(get_cut(line[i],tr[j][k],tr[j][k+1]))
            cut.pb(cp);
            sort(cut.begin(),cut.end());
            cut.resize(unique(cut.begin(),cut.end())-cut.begin());
            if(cut.size()==2)
            {
                mark.pb(mp(cut[0].x,0));
                mark.pb(mp(cut[1].x,1));
            }
        }
        for(j=0;j<seg.size();j++)
        if(line[i]==get_L(seg[j].a,seg[j].b))
        {
            double s=min(seg[j].a.x,seg[j].b.x);
            double e=max(seg[j].a.x,seg[j].b.x);
            mark.pb(mp(s,2));
            mark.pb(mp(e,3));
        }
        sort(mark.begin(),mark.end());
        int in=0,ct=0;
        double last=mark[0].first;
        for(j=0;j<mark.size();j++)
        {
            double y0=line[i].k*last+line[i].b;
            double y1=line[i].k*mark[j].first+line[i].b;
            if(!in&&ct)res+=(y0+y1)*(mark[j].first-last)/2.0;//,printf("%.1lf %.1lf %.1lf\n",y0,y1,mark[j].first-last);
            last=mark[j].first;
            if(mark[j].second==0)in++;
            if(mark[j].second==1)in--;
            if(mark[j].second==2)ct++;
            if(mark[j].second==3)ct--;
        }
    }
    return res;
}
double sol()
{
    int i,j;
    P A,B;
    for(i=1;i<=n;i++)
    {
        for(j=0;j<m;j++)
        {
            A=tr[i][j];
            B=tr[i][j+1];  
            if(sgn(A.x-B.x)>0)seg1.pb(SEG(A,B));
            if(sgn(A.x-B.x)<0)seg2.pb(SEG(A,B));
        }
    }
    return cal(seg1)-cal(seg2);
}
int main()
{
    init();
    printf("%.2lf\n",sol());
}
    \end{lstlisting}
