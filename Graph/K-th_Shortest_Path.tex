\subsection{第K短路}
A* + Dijstra-第K短路问题\\
估值函数: f = g + h\\
对于估值函数中的h\\
对于无向图,我们只需对原图做一次在 终点 的 dijstra ,就可以得到所有点到终点的最短距离\\
对于有向图,我们可以吧原图反向建边,得到一个新图,再在新图中对 终点 做一次 dijstra ,这样就可以求出各点到终点的最短距离\\
From s to t的第k短路\\
    \begin{lstlisting}[language=c++]
typedef struct Status {
    int pos, len, h;
    bool operator < (const Status &cmp) const {
        if(cmp.len + cmp.h == len + h)  return cmp.len < len;
        else                            return cmp.len + cmp.h < len + h;
    }
} Status;

typedef struct Edge {
    int v, next, w;
} Edge;

Edge node[MAXM + MAXN], in_node[MAXM + MAXN];
int dis[MAXN];
bool vis[MAXN];
int n, m, ednum, inednum;

int kth_path(int s, int t, int k) {
    if(dis[s] == INF)   return -1;
    if(s == t)  ++ k;
    priority_queue <Status> q;
    Status now;
    now.pos = s, now.len = 0, now.h = dis[s];
    q.push(now);
    int cnt = 0;
    while(!q.empty()) {
        Status top = q.top(); q.pop();
        if(top.pos == t) ++ cnt;
        if(cnt == k)    return top.len;
        for(int i = node[top.pos].next; i != -1; i = node[i].next) {
            Status tmp;
            tmp.pos = node[i].v, tmp.len = node[i].w + top.len, tmp.h = dis[node[i].v];
            q.push(tmp);
        }
    }
    return -1;
}
    \end{lstlisting}
