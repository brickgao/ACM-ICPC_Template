\subsection{第K值}
    \begin{lstlisting}[language=c++]
#include <cmath>
#include <cstdio>
#include <cstdlib>
#include <cstring>
#include <iostream>
#include <algorithm>
using namespace std;
struct SegT
{
    int l,r,lc,rc,sum;
}tree[233333*9];
int val[233333],s[233333],tot,ROOT[233333];
int maketree(int l,int r)
{
    int k=++tot;
    tree[tot].l=l,tree[tot].r=r;
    if(l==r)
        return k;
    int mid=(l+r)>>1;    
    tree[k].lc=maketree(l,mid);
    tree[k].rc=maketree(mid+1,r);
    return k;
}
int change(int pre,int pos)
{
    int k=++tot;
    tree[k]=tree[pre];
    tree[k].sum++;
    if(tree[k].l==tree[k].r)
        return k;
    int mid=(tree[k].l+tree[k].r)>>1;
    if(pos<=mid) tree[k].lc=change(tree[pre].lc,pos),tree[k].rc=tree[pre].rc;
    else tree[k].rc=change(tree[pre].rc,pos),tree[k].lc=tree[pre].lc;
    return k;
}
int find(int now,int pre,int k)
{
    if(tree[now].l==tree[now].r)
        return tree[now].l;
    int tmp=tree[tree[now].lc].sum-tree[tree[pre].lc].sum;
    if(tmp>=k)
        return find(tree[now].lc,tree[pre].lc,k);
    return find(tree[now].rc,tree[pre].rc,k-tmp);
}
int main() {
    int n,m;
    scanf("%d %d",&n,&m);
    for(int i=1;i<=n;i++)
        scanf("%d",&val[i]);
    memcpy(s,val,sizeof(val));
    sort(s+1,s+1+n);
    ROOT[0]=maketree(1,n);
    for(int i=1;i<=n;i++)
        ROOT[i]=change(ROOT[i-1],lower_bound(s+1,s+1+n,val[i])-s);
    for(int i=1;i<=m;i++)
    {
        int a,b,c;
        scanf("%d %d %d",&a,&b,&c);
        printf("%d\n",s[find(ROOT[b],ROOT[a-1],c)]);
    }
    return 0;
}
    \end{lstlisting}
