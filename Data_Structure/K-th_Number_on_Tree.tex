\subsection{树上第K值}
    \begin{lstlisting}[language=c++]
#include <cmath>
#include <cstdio>
#include <cstdlib>
#include <cstring>
#include <iostream>
#include <algorithm>
#include <map>
#include <set>
#include <vector>
#include <string>
#include <queue>
using namespace std;
#define out(v) cerr << #v << ": " << (v) << endl
#define SZ(v) ((int)(v).size())
#define pb push_back
const int maxint = -1u>>1;
template <class T> bool get_max(T& a, const T &b) {return b > a? a = b, 1: 0;}
template <class T> bool get_min(T& a, const T &b) {return b < a? a = b, 1: 0;}
struct SegT
{
    int l,r,sum,lc,rc;
}tree[2333333];
int last[233333],pre[233333],son[233333],tot,MAX;
int n,m,f[233333][21];
int root[233333],dist[233333],head,tail,que[233333],val[233333],s[233333];
void add(int a,int b)
{
    tot++;
    son[tot]=b;
    pre[tot]=last[a];
    last[a]=tot;
}
void bfs()
{
    dist[1]=1;
    que[tail++]=1;
    while(head<tail)
    {
        int p=que[head++];
        for(int x=last[p];x!=0;x=pre[x])
        {
            if(f[p][0]!=son[x])
            {
                dist[son[x]]=dist[p]+1;
                f[son[x]][0]=p;
                que[tail++]=son[x];
                for(int j=1;j<18;j++)
                    f[son[x]][j]=f[f[son[x]][j-1]][j-1];
            }    
        }    
    }
}
int lca(int a,int b)
{
    if(dist[a]>dist[b]) swap(a,b);
    for(int i=17;i>=0;i--)
    {
        if(dist[f[b][i]]>=dist[a]) b=f[b][i];
        if(a==b) return a;    
    }    
    for(int i=17;i>=0;i--)
        if(f[a][i]!=f[b][i]) a=f[a][i],b=f[b][i];
    return f[a][0];
}

int maketree(int l,int r)
{
    int k=++tot;
    tree[k].l=l,tree[k].r=r;
    if(l==r)
        return k;
    int mid=(l+r)>>1;
    tree[k].lc=maketree(l,mid);    
    tree[k].rc=maketree(mid+1,r);
    return k; 
}
int change(int pr,int pos)
{
    int k=++tot;
    tree[k]=tree[pr];
    tree[k].sum++;
    if(tree[k].l==tree[k].r)
        return k;
    int mid=(tree[k].l+tree[k].r)>>1;
    if(pos<=mid) 
    {
        tree[k].lc=change(tree[pr].lc,pos);
        tree[k].rc=tree[pr].rc;
    }
    else 
    {
        tree[k].lc=tree[pr].lc;
        tree[k].rc=change(tree[pr].rc,pos);
    }
    return k;
}
int find(int a,int b,int p,int pf,int k)
{
    int l=1,r=MAX;
    while(l!=r)
    {
        int mid=(l+r)>>1; 
        int sum=tree[tree[a].lc].sum+tree[tree[b].lc].sum-tree[tree[p].lc].sum-tree[tree[pf].lc].sum;
        if(sum>=k) a=tree[a].lc,b=tree[b].lc,p=tree[p].lc,pf=tree[pf].lc,r=mid;
        else a=tree[a].rc,b=tree[b].rc,p=tree[p].rc,pf=tree[pf].rc,k-=sum,l=mid+1;
    }
    return l;
}
int main() {
    scanf("%d %d",&n,&m);
    for(int i=1;i<=n;i++)
        scanf("%d",&val[i]),s[i]=val[i];
    sort(s+1,s+1+n);
    MAX=unique(s+1,s+1+n)-s-1;
    for(int i=1;i<=n;i++)
        val[i]=lower_bound(s+1,s+1+MAX,val[i])-s;
    for(int i=1;i<=n-1;i++)
    {
        int a,b;
        scanf("%d %d",&a,&b);
        add(a,b);
        add(b,a);
    }
    bfs();
    tot=0;
    root[0]=maketree(1,MAX);
    for(int i=0;i<tail;i++)
        root[que[i]]=change(root[f[que[i]][0]],val[que[i]]);
    for(int i=1;i<=m;i++)
    {
        int a,b,c;
        scanf("%d %d %d",&a,&b,&c);
        int p=lca(a,b);
        int pf=f[p][0];
        printf("%d\n",s[find(root[a],root[b],root[p],root[pf],c)]);
    }
    return 0;
}
    \end{lstlisting}
